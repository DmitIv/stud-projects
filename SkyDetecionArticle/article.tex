% This is samplepaper.tex, a sample chapter demonstrating the
% LLNCS macro package for Springer Computer Science proceedings;
% Version 2.20 of 2017/10/04
%
\documentclass[runningheads]{llncs}


%
\usepackage{threeparttable}
\usepackage[dvipsnames]{xcolor}
\usepackage{amsmath}
\usepackage{graphicx}
\usepackage[utf8]{inputenc}
\usepackage[russian]{babel}
\usepackage{array}
\usepackage{multirow}
\usepackage{float}
% Used for displaying a sample figure. If possible, figure files should
% be included in EPS format.
%
% If you use the hyperref package, please uncomment the following line
% to display URLs in blue roman font according to Springer's eBook style:
% \renewcommand\UrlFont{\color{blue}\rmfamily}
\newcommand\MyBox[2]{
  \fbox{\lower0.75cm
    \vbox to 1.7cm{\vfil
      \hbox to 1.7cm{\hfil\parbox{1.4cm}{#1\\#2}\hfil}
      \vfil}%
  }%
}


\begin{document}
%
\title{АВТОМАТИЧЕСКАЯ СЕГМЕНТАЦИЯ НЕБА НА ИЗОБРАЖЕНИЯХ}
%
%\titlerunning{Abbreviated paper title}
% If the paper title is too long for the running head, you can set
% an abbreviated paper title here
%
\author{Д. И. Ивахненко\inst{1}\orcidID{0000-0003-1493-5192} \and
M.V.Yurushkin\inst{1}\orcidID{0000-0003-2477-0459}}
%
% First names are abbreviated in the running head.
% If there are more than two authors, 'et al.' is used.
%
\institute{I.I. Vorovich Institute of Mathematics, Mechanics, and Computer Science, \\ 8A Milchakova St., Rostov-on-Don, 344090, Russian Federation}
%
\maketitle              % typeset the header of the contribution
%
\begin{abstract}
Данная работа посвящена вопросу детекции участков неба на
произвольном входном изображении.
Входное изображение можно рассматривать как набор из пикселей, 
в котором для каждого элемента необходимо указать вероятность
его принадлежности к определенному классу, в данном контексте 
одному - классу неба. При таком подходе данная проблема сводится
к задаче сегментации изображений. 
В ходе работы был сформирован датасет, аналогичный SkyFinder; разработан автоматический подход сегментации 
неба на изображениях, устойчивый к разнообразным условиям освещенности, погодным явлениям и т.д. 
Решение строится на применении глубоких нейронных сетей с
дальнейшей обработкой результатов работы сети по средствам алгоритмов 
компьютерного зрения. Использовались различные архитектуры 
сети, сравнивалась их эффективность.
Были получены высокие показатели метрик accuracy и F1-score на тестовой выборке.

\keywords{Сегментация  \and Выделение неба \and Сверточные сети.}
\end{abstract}
%
%
%

\section{Introduction}

Нами была поставлена задача разработки алгоритма, способного для произвольного входного
изображения, сделанного на камеру мобильного устройства, опеределить 
область неба. Подобного рода решения могут быть использованы в конечных
пользовательских продуктах, направленных на работу с фото или видео, 
применяться для получения дополнительных данных для алгоритмов 
обработки изображений\cite{ind_outd}.
В качестве результата работы такого алгоритма
рассматривается бинарная маска - матрица, каждый 
элемент которой принимает значение в зависимости от определенного 
для него класа: 1 - класс неба, 0 - класс остальных объектов.
Оценка качества получаемых ответов проводится путем вычисления
метрик accuracy и F1-score. 
В рамках этой работы мы освещаем применение fully convolutional networks к задачи сегментации изображения (пункт 4.2),
аугментацию данных как способ искусственно разнообразить имеющийся датасет и предотвратить переобучение сети на конкретных изображениях (пункт 4.1), использование алгоритмов компьютерного зрения для корректировки точности сегментации, полученной после работы 
сети (пункт 4.3). В разделе 5 описаны используемые нами метрики и функция потерь, приведены результаты обучения для различных подходов к построению сети и методов постобработки.

\section{Background and related work}

Для сегментации изображения существуют подходы, 
не предполагающие использования нейронных сетей\cite{seg1}\cite{seg2}\cite{seg3}, но в нашей 
задаче применение таких решений оказалось невозможным. Методы 
нахождения порога бинаризации и алгоритмы, основанные на
использовании предикатов для разделения пикселей по классам, демонстрируют 
удовлетворительные результаты только на конкретных
примерах, имея плохую способность к обобщению: так бинаризация
по методу Отсу на произвольном входном изображении оказывается
недостаточно точная, выделяя, помимо верных, все достаточно яркие
и, тем самым, оказавшиеся выше порога пиксели. В случае алгоритмов, 
основанных на проверке на близость к цвету, необходимо задавать 
для каждого отдельно взятого изображения его цветовой центр -
значение в RGB представлении для цвета, с которым будет идти сравнение\cite{eff_app}.
Отметим здесь, что можно
рассмотреть и другое решение: искать границу на изображении между 
землей и небом\cite{ef_sky_detect}

\section{Datasets}

Нами был подготовлен набор из 1270 пар изображений:
оригинальная фотография и соотвествующая ей бинаризированная маска, на которой белым цветом была отмечена область неба, черным - остальная часть изображения. Данные были разбиты на три группы:
обучающую, тестовую и валидационную. В обучающую вошли 1100
элементов, в тестовую 100 и 70 в валидационную. На обучающей выборке 
проводилась тренировка модели, на тестовой подсчитывались
метрики в конце каждой эпохи, элементы из валдиационной выборки
использовались для оценки качества различных моделей и подходов к
обработке маски. В выборках присутсвуют изображения с различной
погодой, временем суток, ландшафтом и яркостью цветов, что снижает 
вероятность переобучения модели по какому-то из признаков.
Особую сложность в рамках данной задачи представляет высокое 
разнообразие состояний неба на входных изображениях: на
одном изображении небо может быть выражено широким спектром
цветов, его отдельные участки могут быть перекрыты объектами, к
нему не относящимися и пр..

\section{Our approach}
С учетом вышесказанного делается предположение о необходимости 
использования глубоких сверточных сетей, которые демонстрируют 
высокую точность в задачах обработки изображения\cite{fcn}, в качестве решения. 
За основу была взята U-net топология сети, были
опробованы различные способы построения енкодер-части модели
(the contracting path), участвующей в извлечении характеристик из
входных данных.
Подобный подход был продемонстрирован Cecilia La Place, Aisha
Urooj Khan и Ali Borji в их работе Segmenting Sky Pixels in Images\cite{seg_sky}.
В рамках того исследования была использована топология RefineNet
без последующей обработки выходной маски, обучения проводилось
на датасете SkyFinder\cite{sky_seg_wild}.

\subsection{Preprocessing}

Перед тренировкой сверточной сети в нашем решении производится аугментация данных для снижения вероятности переобучения на изображениях тренировачной выборки, проводимая следующим образом:
\begin{itemize}
\item Увеличение, уменьшение яркости.
\item Поворот изображения на угол до 45 градусов.
\item Отражение изображения относительно вертикальной и горизонтальной осей.
\item Масштабирование изображения.
\item Отражение границ изображения.
\end{itemize}
Последний способ описан в \cite{unet}: у изображения берутся отступы  от верхней и нижней границ, отражаются и склеиваются с ним; затем получившееся изображение масштабируется к исходным размерам. Примеры работы метода отражения продемонстрированы на рис. 1 и рис. 2. Данный способ позволяет {сохранить пространственную 
информацию, которая может теряться на сверточных
слоях, для пикселей, находящихся вблизи границы изображения.

\begin{figure}[H]
  \centering
  \begin{minipage}[b]{0.4\textwidth}
    \includegraphics[width=\textwidth]{before_mir.png}
    \caption{Оригинальное изображение.}
  \end{minipage}
  \hfill
  \begin{minipage}[b]{0.4\textwidth}
    \includegraphics[width=\textwidth]{after_mir.png}
    \caption{Изображение с отраженными границами.}
  \end{minipage}
\end{figure}

\subsection{Deep model architecture}

В решении использовалась архитектура, представленная
Olaf Ronneberger, Philipp Fischer и Thomas Brox в работе U-Net:
Convolutional Networks for BiomedicalImage Segmentation\cite{unet}. Данный
подход подразумевает отражение сверточной сети с заменой операций 
понижения размерности путем взятия максимального значения
по окну на операции повышения размерности. Часть итоговой сети с
операциями понижения размерности (енкодер-часть) извлекает информацию 
из входных данных по средствам применения сверточных
слоев, часть с операциями повышения размерности (декодер-часть),
используя выходную с енкодер-части информацию строит итоговую
маску. Характеристики, получаемые на этапе извлечения, комбинируются 
с результатами работы декодер-части. На рис.~\ref{fig1} представлена
полная архитектура сети.

\begin{figure}[H]
\includegraphics[width=\textwidth]{unet.png}
\caption{Unet архитектура} \label{fig1}
\end{figure}

Для улучшения показателей качества сети (метрик accuracy и
F1-score) енкодер-часть была заменена на схожие по функциональности 
сети: ResNet50\cite{resnet}, ResNet101\cite{resnet}, MobileNetV2\cite{mobnet}. Основная идея такой замены следующая: использовать сеть, предобученую на большем датасете, чем имеется в рамках этой задачи,
и демонстрирующую лучшие результаты в извлечении признаков, чем базовый подход в Unet топологии.

\subsection{Postporocessing}
Результатом работы сети является матрица вероятностей принадлежности каждого
пикселя к целевому классу. Такой результат необходимо
бинаризировать для того, чтобы строго выделить небо на изображении. Ниже приведены три метода, протестированные в рамках решения.

\subsubsection{Бинаризация округлением}
Первый метод бинаризации заключался в округлении 
к ближайшему целому числу. Он показал, как видно из табл.\ref{tab3}, наименьшее значения метрики качества как для базовой Unet топологии, так и для ее модификации. 
\subsubsection{Бинаризация методом Отсу} 
Результат лучше продемонстрировал метод Отсу для бинаризации изображения, его использование улучшило качество модели в сравнении с бинаризацией округлением. Бинаризация по Отсу минимизирует внутриклассовую дисперсию.

\subsubsection{Бинаризация с восстановлением границ} 
Этот метод совмещает в себе последовательно второй метод и исправление неточности классификации пикселей. Исправление неточности соврешается следующим образом: к бинаризированной маске применяется алгроитм FindContours из состава библиотеки OpenCV для нахождения замкнутых контуров, принадлежащих к классу неба; затем отбираются те из них, чья площадь меньше заданного значения (значение подбиралось опытным путем на примерах), и помечаются как класс остальных объектов; последним шагом восстанавливается точность границы между двумя классами.

Восстановление точности границы между классами представляет  собой сложение маски, полученной методом бинаризации по Отсу, и маски после отбора контуров. На первой граница классов более точная, 
на второй отсутствуют ложно классифицированные объекты.
Положим, $O$ - матрица-маска, полученная методом Отсу, элементы 
равны $0$ или ${\text{1}}$; ${\text{A}}$ - матрица-маска, полученная отбором контуров 
после метода Отсу, элементы равны ${\text{0}}$ или ${\text{1}}$. Тогда ${\text{O'}}$ - матрица,
для элементов которой справедливо: ${o'_{ij} = 0}$, если ${o_{ij} = 1}$ ${o_{ij} \in \text{O}}$, иначе
${o'_{ij} = 1}$(1). Аналогично построим ${\text{A'}}$ для ${\text{A}}$. Сложением матриц ${\text{O'}}$ и ${\text{A'}}$
получаем матрицу ${\text{B'}}$ . Матрица-маска ${\text{B'}}$, для которой справедливо (1)
в отношении ${\text{B}}$ , обладает границей с восстановленной точностью.

\begin{figure}[H]
  \centering
  \begin{minipage}[b]{0.4\textwidth}
    \includegraphics[width=\textwidth]{before_restore.png}
    \caption{После отбора контуров.}
    \label{fig3}
  \end{minipage}
  \hfill
  \begin{minipage}[b]{0.4\textwidth}
    \includegraphics[width=\textwidth]{after_restore.png}
    \caption{После восстановления.}
    \label{fig4}
  \end{minipage}
\end{figure}

На рис.4 рис.5 приведено сравнение двух масок: до и после восстановления 
точности. Детализация границы выше на правом изображении.

\section{Experimental results}

Обучение модели происходило на протяжении ста эпох, по окончанию 
каждой замерялась метрика accuracy для валидационной выборки 
из исходного датасета. По завершению эпох на тестовой выобрке 
замерялись метрики accuracy и F1-score.

Метрики accuracy и F1-score вычисляются с помощью матрицы
ошибок (табл.~\ref{tab1}). Результаты обучения приведены
в табл.~\ref{tab2} и табл.~\ref{tab3}.

\noindent
\renewcommand\arraystretch{1.5}
\setlength\tabcolsep{0pt}
\begin{table}[H]
\centering
\caption{Confusion matrix}\label{tab1}
\begin{tabular}
{c >{\bfseries}r @{\hspace{0.7em}}c @{\hspace{0.4em}}c @{\hspace{0.7em}}l}
  \multirow{10}{*}{\parbox{1.1cm}{\bfseries\raggedleft actual\\ value}} & 
    & \multicolumn{2}{c}{\bfseries Prediction outcome} & \\
  & & \bfseries p & \bfseries n & \bfseries total \\
  & p$'$ & \MyBox{True}{Positive} & \MyBox{False}{Negative} & P$'$ \\[2.4em]
  & n$'$ & \MyBox{False}{Positive} & \MyBox{True}{Negative} & N$'$ \\
  & total & P & N &
\end{tabular}
\end{table}

$$
F_{1}=2 \cdot \frac{\text{Precision} \times \text{Recall}}{\text{Precision} + \text{Recall}}
$$
, где
\[
Precision=\frac{\text{TP}}{\text{TP + FP} }
\]
и
\[
Recall=\frac{\text{TP}}{\text{TP + FN} };
\]
\[
Accuracy=\frac{\text{TP + TN}}{\text{TP + FP + TN + FN} }.
\]

В качестве функции потерь использовалась усредненная по размеру 
входного батча логистическая функция ошибки.

\[
H_{p}(q)= -\frac{1}{N} \cdot \sum_{i=1}^{N}(y_{i} \cdot \log{(p(y_{i}))} + (1 - y_{i}) \cdot \log{(1 - p(y_{i}))})
\]

\begin{table}[H]
    \begin{minipage}{.45\linewidth}
      \centering
\caption{Сравнение метрики F1-score различных енкодер-частей}
\label{tab2}
       \begin{tabular}{|l|r|}
\hline
Модель &  F1-score \\
\hline
Unet &  0.9729 \\ 
ResNet50 + Unet & 0.9891\\ 
ResNet101 + Unet & 0.9895\\ 
MobileNetV2 + Unet & 0.9789\\ \hline
\end{tabular}

    \end{minipage}%
	\hspace{2em}
    \begin{minipage}{.45\linewidth}
      \centering
      \caption{Сравнение качества по метрике F1-score для различных
подходов к обработке маски для различных моделей}
\label{tab3}
        \begin{tabular}{|l|c|r|}
\hline
Модель & Способ постобработки & F1-score\\
\hline
Unet & Бинаризация округлением & 0.9729 \\
Unet & Бинаризация по методу Отсу & 0.9736 \\ 
Unet & Бинаризация с восстановлением границ &  0.9767 \\ 
Unet+ResNet50 & Бинаризация округлением & 0.9891 \\ 
Unet+ResNet50 & Бинаризация по методу Отсу & 0.9892 \\
Unet+ResNet50 & Бинаризация с восстановлением границ & 0.9901 \\ \hline
\end{tabular}
    \end{minipage} 
\end{table}


\begin{figure}[H]
\includegraphics[width=0.8\textwidth]{fig.png}
\caption{Сравнение значений функции потери для разных моделей 
на обучающей выборке по эпохам. По горизонтали
на графике отмерены эпохи, по вертикали - значения
функции потерь. Нижняя линия - Unet+ResNet50, верхняя 
- базовый Unet} \label{fig2}
\end{figure}

Лучший результат демонстрирует сочетание ResNet101 и Unet с постобработкой по методу 3, следующим по качеству выступает сочетание ResNet50, Unet и метода 3. Отметим, что использование ResNet101 вместо ResNet50 не дает ощутимого прироста качества при значительном увелечении объема сети.


\section{Conclusion}
В рамках данной работы был разработан алгоритм автоматической 
детекции неба на произвольном входном изображении. В основу
подхода легло использование глубоких сверточных сетей с последующей 
обработкой маски алгоритмами из области компьютерного зрения. 
Система демонстрирует высокое качество по метрике F1-score и
accuracy на валидационных и тестовых выборках, что говорит о возможности 
ее применимости на произвольных данных. К недостаткам
можно отнести низкую точность на изображениях, сделанных в ночное 
время или содержащих смог, туман.
Примеры работы алгоритма (красным на изображении выделена 
область неба):

\begin{figure}[H]
\begin{minipage}[b]{0.4\textwidth}
    \centering\includegraphics[width=5cm]{example2.png}
    \caption{Пример 1.}
\end{minipage}
\begin{minipage}[b]{0.4\textwidth}
    \centering\includegraphics[width=5cm]{example3.png}
    \caption{Пример 2.}
\end{minipage}
\begin{minipage}[b]{0.4\textwidth}
    \centering\includegraphics[width=5cm]{example1.png}
    \caption{Пример 3.}
\end{minipage}
\end{figure}
%
% ---- Bibliography ----
%
% BibTeX users should specify bibliography style 'splncs04'.
% References will then be sorted and formatted in the correct style.
%
% \bibliographystyle{splncs04}
% \bibliography{mybibliography}
%
\section{References}
\begin{thebibliography}{8}
\bibitem{ind_outd}
 Jiebo Luo, A. Savakis : Indoor vs outdoor classification of consumer photographs using low-level and semantic features. 
 Published in: Proceedings 2001 International Conference on Image Processing (Cat. No.01CH37205), Oct. 2001 
IEEE(2001). \doi{10.1109/ICIP.2001.958601}

\bibitem{seg1}
 K. J. Batenburg, J. Sijbers : Optimal Threshold Selection for Tomogram Segmentation by Projection Distance Minimization. 
 Published in: IEEE Transactions on Medical Imaging, Volume: 28 , Issue: 5, pp  676 -- 686, May 2009 

\bibitem{seg2}
 Alireza Kashanipour, Narges Shamshiri Milani, Amir Reza Kashanipour : Robust Color Classification Using Fuzzy Rule-Based Particle Swarm Optimization
Published in: 2008 Congress on Image and Signal Processing 
July 2008
\doi{10.1109/CISP.2008.770}

\bibitem{seg3}
Dumitru Dan Burdescu, Marius Brezovan, Liana Stanescu, Cosmin Stoica Spahiu: A Spatial Segmentation Method
International Journal of Computer Science and Applications
Vol. 11, No. 1, pp. 75 -- 100, 2014 

\bibitem{eff_app}
Irfanullah, Kamal Haider, Qasim Sattar, Sadaqat-ur-Rehman, Amjad Ali: An Efficient Approach for Sky Detection. 
IJCSI International Journal of Computer Science Issues, Vol. 10, Issue 4, No 1, July 2013
, pp. 223--226.

\bibitem{ef_sky_detect}
Chi-Wei Wang, Jian-Jiun Ding, Po-Jen Chen: An Efficient Sky Detection Algorithm Based on Hybrid Probability Model. 
Proceedings of APSIPA Annual Summit and Conference 2015, pp. 919--922, Dec. 2015.

\bibitem{fcn}
Jonathan Long, Evan Shelhamer, Trevor Darrell: Fully Convolutional Networks for Semantic Segmentation. 
IEEE Transactions on Pattern Analysis and Machine Intelligence
Volume 39 Issue 4, April 2017
Page 640-651 

\bibitem{seg_sky}
Cecilia La Place, Aisha Urooj Khan, Ali Borji: Segmenting Sky Pixels in Images
, {\tt  arXiv:1712.09161v2 [cs.CV]}
8 Jan 2018

\bibitem{sky_seg_wild}
Radu P. Mihail, Valdosta State University, Scott Workman, Zach Bessinger, Nathan Jacobs: Sky segmentation in the wild: An empirical study
Published in: 2016 IEEE Winter Conference on Applications of Computer Vision (WACV), March 2016, IEEE(2016).
\doi{10.1109/WACV.2016.7477637}

\bibitem{unet}
Olaf Ronneberger, Philipp Fischer, Thomas Brox: U-Net: Convolutional Networks for BiomedicalImage Segmentation
, {\tt  arXiv:1505.04597v1  [cs.CV]}
18 May 2015

\bibitem{resnet}
Kaiming He, Xiangyu Zhang, Shaoqing Ren, Jian Sun: Deep Residual Learning for Image Recognition
, {\tt  arXiv:1512.03385v1  [cs.CV]}
10 Dec 2015

\bibitem{mobnet}
Mark Sandler, Andrew Howard, Menglong Zhu, Andrey Zhmoginov, Liang-Chieh Chen: MobileNetV2: Inverted Residuals and Linear Bottlenecks
, {\tt arXiv:1801.04381v4  [cs.CV]}
21 Mar 2019
\end{thebibliography}
\end{document}
