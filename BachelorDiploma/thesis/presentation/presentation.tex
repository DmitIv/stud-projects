%& -job-name=presentation
\documentclass[
    10pt,           % размер шрифта
    aspectratio=43  % соотношение сторон слайдов
]{beamer}

\def\jobname{presentation}
    
%% Tема презентации
\usetheme[]{metropolis}         
\usefonttheme[]{professionalfonts}  % запрещаем beamer'у перезаписывать мат. шрифты

%% Пакеты и их настройки
\usepackage[]{fontspec}         % загрузка шрифтов, работа с кодировкой и пр.
\usepackage[]{polyglossia}      % переносы слов
\usepackage[]{microtype}        % различные исправления типографии
\usepackage[]{unicode-math}     % использование математических символов юникода
\usepackage[]{graphicx}
\usepackage[]{subfig}
\usepackage[]{float}
\usepackage[]{enumitem}
\usepackage[]{tabularx}
\usepackage[]{booktabs}
\usepackage{csquotes} % big quotations
\usepackage[]{caption}
\usepackage[]{subcaption}

\frenchspacing                  %% убираем лишние отступы после точек

%% Перенос слов
\binoppenalty       = 10000 %% Запрет переносов строк в формулах
\relpenalty         = 10000
\pretolerance       = 5000  %% Настройки переноса
\tolerance          = 9000  %% Настройки переноса
\emergencystretch   = 0pt   %% Запрещаем выход за границы
\righthyphenmin     = 2     %% целое число, равное наименьшему количеству букв в слове, которые можно переносить на следующую строку
\lefthyphenmin      = 2
\hyphenpenalty      = 500
\clubpenalty        = 10000 %% Запрет разрывов страниц после первой
\widowpenalty       = 10000 %% и перед предпоследней строкой абзаца

%% Язык
\setmainlanguage[spelling=modern]{russian}
\setotherlanguage{english}

%% Шрифты
\defaultfontfeatures{
    Ligatures  = {TeX, Common},
    Mapping    = tex-text
}

%% Main font
\setmainfont[
    Path            = ./fonts/,
    Extension       = .otf,
    UprightFont     = *-Regular,
    BoldFont        = *-Bold,
    ItalicFont      = *-Italic,
    BoldItalicFont  = *-BoldItalic
]{FreeSerif}
\newfontfamily\cyrillicfont[
    Path            = ./fonts/,
    Extension       = .otf,
    UprightFont     = *-Regular,
    BoldFont        = *-Bold,
    ItalicFont      = *-Italic,
    BoldItalicFont  = *-BoldItalic
]{FreeSerif}

%% Roman font
\setromanfont[
    Path            = ./fonts/,
    Extension       = .otf,
    UprightFont     = *-Regular,
    BoldFont        = *-Bold,
    ItalicFont      = *-Italic,
    BoldItalicFont  = *-BoldItalic
]{FreeSerif}
\newfontfamily\cyrillicfontrm[
    Path            = ./fonts/,
    Extension       = .otf,
    UprightFont     = *-Regular,
    BoldFont        = *-Bold,
    ItalicFont      = *-Italic,
    BoldItalicFont  = *-BoldItalic
]{FreeSerif}

%% Sans font
\setsansfont[
    Path            = ./fonts/,
    Extension       = .otf,
    UprightFont     = *-Regular,
    BoldFont        = *-Bold,
    ItalicFont      = *-Oblique,
    BoldItalicFont  = *-BoldOblique
]{FreeSans}
\newfontfamily\cyrillicfontsf[
    Path            = ./fonts/,
    Extension       = .otf,
    UprightFont     = *-Regular,
    BoldFont        = *-Bold,
    ItalicFont      = *-Oblique,
    BoldItalicFont  = *-BoldOblique
]{FreeSans}

%% Mono font
\setmonofont[
    Path            = ./fonts/,
    Extension       = .otf,
    UprightFont     = *-Regular,
    BoldFont        = *-Bold,
    ItalicFont      = *-Oblique,
    BoldItalicFont  = *-BoldOblique
]{FreeMono}
\newfontfamily\cyrillicfonttt[
    Path            = ./fonts/,
    Extension       = .otf,
    UprightFont     = *-Regular,
    BoldFont        = *-Bold,
    ItalicFont      = *-Oblique,
    BoldItalicFont  = *-BoldOblique
]{FreeMono}

%% Math font
\setmathfont[
    Path        = ./fonts/,
    Extension   = .otf,
    BoldFont    = *Bold
]{XITS-Math}


%% --------------------------------------------------------------
%% Настройки пользователя
%% --------------------------------------------------------------

\graphicspath{{/home/dmitri/PycharmProjects/BachelorDiploma/thesis/mmcs_sfedu_thesis/img/}}
%% Параметры темы
%% http://mirrors.mi.ras.ru/CTAN/macros/latex/contrib/beamer-contrib/themes/metropolis/doc/metropolistheme.pdf
\metroset{
    numbering   = fraction, % нумерация слайдов
    progressbar = frametitle, % информация о прогрессе
    background  = light % светлая тема
}


\title{Семантический анализ фотографий с помощью глубоких нейронных сетей}
%\titlegraphic{\includegraphics[height=1.0cm]{sfedu_logo.png}}
%\subtitle{}

\author{
Выпускная квалификационная работы
\vskip 1em
02.03.02 - Фундаментальная информатика и информационные технологии
\vskip 1em 
Выполнил:\\студент 4 курса Ивахненко Дмитрий Игоревич
\vskip 1em 
Научный руководитель:\\ к.\,ф.-м.\,н., ст.преп. М.\,В. Юрушкин
\vskip 1em
}

\date{24 июня 2020 г.}

\institute{Институт ММиКН им. И.И. Воровича, Южный Федеральный Университет}
